\documentclass[bachelor]{seuthesis} % 本科
% \documentclass[master]{seuthesis} % 硕士
% \documentclass[doctor]{seuthesis} % 博士
% \documentclass[engineering]{seuthesis} % 工程硕士
\usepackage{CJK,CJKnumb}
\usepackage{amsmath}
\usepackage{algorithm}
\usepackage{algorithmicx}
\usepackage{algpseudocode}
\usepackage{indentfirst}
\usepackage{graphicx}
\floatname{algorithm}{算法}
\renewcommand{\algorithmicrequire}{\textbf{输入:}}
\renewcommand{\algorithmicensure}{\textbf{输出:}}
 % 这里是导言区

\begin{document}
\categorynumber{000} % 分类采用《中国图书资料分类法》
\UDC{000}            %《国际十进分类法UDC》的类号
\secretlevel{公开}    %学位论文密级分为"公开"、"内部"、"秘密"和"机密"四种
\studentid{04013239}   %学号要完整,前面的零不能省略。

\title{无线通信网络中基于学习的分布式缓存方案研究}{}{}{}
\author{王驭扬}{Yuyang Wang}
\advisor{杨绿溪}{教授}{Advisor:Lvxi Yang}{Prof.}
%\coadvisor{副导师}{副教授}{Co-advisor's Name}{Associate Prof.} % 没有

% \degree{工学硕士} % 详细学位名称
\major[12em]{信息工程}
\defenddate{答辩日期}
\authorizedate{学位授予日期}
\department{信息科学与工程}{department name}
\duration{2016.12—2017.6}
\address{无线谷A5楼2层}
\thanks{本论文获国家XXX计划项目(2012AA00A00)和国家杰出青年科学基金项目(01234567)资助。}
\maketitle

\begin{abstract}{主动缓存,迁移学习,协同过滤,势博弈}
本文对于无线通信网络中的分布式缓存问题,使用主动缓存技术,通过在网络边缘设备进行缓存来主动地服务可预测的用户需求。\par
无线通信网络中,移动用户的对于流量的需求越来越大,用主动缓存的思想来设计分布式缓存方案,可以有效提高用户数据速率降低回程负载。关于主动缓存的问题我们做了如下的一些研究工作。\par
首先设计了基于文件受欢迎度和用户、文件之间的相关度在非高峰时段主动缓存文件的缓存方案。其中先使用协同过滤和迁移学习来预测和估计文件对用户的受欢迎度矩阵,通过对矩阵奇异值分解目标函数的不断优化提出了优化后的协同过滤算法。接着在此基础上,结合迁移学习,以待解决的缓存问题为目标域,以用户间端到端(D2D)通信等相关的信息交互为源域,通过对源域的先验信息和目标域中稀疏的已知数据信息共同地进行学习来更好地解决目标域的缓存问题,并提出了基于迁移学习的缓存算法。\par
然后讨论了主动缓存中小基站如何进行缓存的问题,并对小基站缓存内容的决策方案进行了优化。从博弈论的角度证明了分布式缓存问题可建立为一个严格势博弈模型,提出了博弈缓存算法来求出博弈的最佳纳什均衡,也即缓存问题的全局最优解。\par
最后仿真并分析了所提出算法的性能,得到了我们的基于迁移学习的缓存算法和分布式博弈缓存算法的性能都要比传统算法好的结论。
\end{abstract}

\begin{englishabstract}{Proactive caching, Transfer learning, Collaborative filtering, Potential game}
In this paper, we use proactive caching technology to solve the distributed caching problem in the wireless communication network, by proactively serving predictable user demands via caching at network edge. \par
In the wireless communication network, mobile users' demand for traffic is growing, and the distributed cache scheme designed by proactive caching can effectively increase the users' data rate and alleviate backhaul congestion. Our work on the issue of proactive caching are shown as follows.\par
Firstly, the proposed caching paradigm proactively caches files at off-peak peroids based on the popularity of the file and the dependencies between the user and the file. We use the collaborative filtering and transfer learning to predict and estimated content popularity matrix, the optimized collaborative filtering algorithm is proposed by optimizing the objective function of matrix singular value decomposition. Then, combined with transfer learning, we set the caching problem as target domain,and the user's deviceto-device(D2D) interactions as source domain. The prior information in the source domain is incorporated in the target domain with sparse data information where the goal is to optimally cache contents, solving the cache problem in the target domain, and a caching algorithm based on transfer learning is proposed. \par
Then, the problem of small base station how to cache content in proactive caching is discussed, and the caching decision scheme of small base station is optimized. From the point of view of game theory, we formulate the distributed caching problem as an exact potential game(EPG), then a game caching algorithm is proposed to find the best Nash equilibrium(NE) of the game, and it is the global optimum solution of the caching problem. \par
Finally, the performance of the proposed algorithms is simulated and analyzed, and we get the conclusion that our caching algorithm based on transfer learning and distributed game caching algorithm are better than classical algorithms.
\end{englishabstract}

\tableofcontents

%\begin{terminology}
%\end{terminology}

\begin{Main} % 开始正文

\chapter{绪论}
\section{研究背景}
随着移动通信、移动智能终端以及社交网络的发展,我们开始越来越多的通过移动终端来体验丰富多样的在线服务,在工作、学习、生活等方面也正逐渐向移动互联网迁徙。在这个互联网快速发展的时代,我们开始注重移动网络带来的服务体验,而这也对快速获取更充实更有质量的内容提出了要求。用户需求层次的提高带动了大量新兴无线通信业务的发展,同时也带来了移动互联网的数据量爆炸式增长。最近的调查表明移动视频的数据流量占据了总移动数据流量的一半左右\cite{cisco2014global},而且这项数据还在急速增长中。\par
移动数据量的增长需要更大的网络容量,搭建小基站、减小与用户之间的距离是一个非常有效的方法。实际上,基站的密集程度已经从之前的按数百平方公里进行部署到目前的几分之一平方米甚至更小,近来也开始在住宅、地铁、企业和热点地区部署中继站、分布式天线和小型蜂窝接入点,这些由运营商或用户部署的网络架构被称为异构网络或小蜂窝网络。通过在局域范围内部署额外的网络节点,使得网络更接近用户终端,可以更有效地改善空间复用和覆盖,提高网络容量和减少网络流量。\par
在这种无线通信网络中,为应对当今移动数据量的爆炸式增长,我们传统的方法是提高频谱资源(带宽),频谱效率(调制、编码、MIMO),或者空间复用(基站密度)。但是在实际应用条件下,这些方法提供的吞吐量增益相当有限,而且实现成本较高。除此之外,未来无线通信网络会更注重用户的体验,为此就需要更小的端到端的时延,但是现有无线通信网络对于每个用户的每个请求都会经过基站进行一次数据传输,长距离的高清视频传输必将导致服务延时较大,同时在终端数量爆炸式增长的情况下,这也会急剧增大基站的网络拥塞进而导致更大的服务时延。所以我们亟待寻找一种合适的方法来缓解这种窘境。无线通信网络缓存应运而生。\par
无线通信网络缓存可以利用网络边缘的各种端设备,如基站、移动终端等来进行内容的缓存。无线通信网络缓存之所以引起学术界及工业界的广泛关注,一方面是因为近年来缓存设备的价格不断下降,而且移动终端的存储空间越来越大;另一方面,无线网络缓存可通过提前缓存内容并利用无线网络边缘内部、边缘设备互相之间良好的短距离协作通信来传输内容,有效降低服务延时,也能减小基站的峰值速率、卸载基站的流量。\par
无线通信网络缓存分为集中式缓存和分布式缓存。考虑到当今无线网络中大规模的移动数据量,越来越多的用户,更加丰富多样的用户需求,我们更迫切的需要性能更好的,无线通信网络中分布式的缓存方案。优越的缓存方案可以更大程度的减少传输重复内容的数量,并且加速实时传输。随着近来学习技术的发展,内容的受欢迎度(即用户对不同内容的需求程度)可以通过跟踪用户的请求频率和分析历史数据来进行预测和估计,然后通过在非高峰时段预先下载受欢迎的内容并提供可预测的高峰时段的用户需求来提高无线网络缓存的效率。因此通过设计合理的基于学习的缓存方案,不仅可以使得繁忙的高峰时段的流量负载问题得到缓解,还可以减少请求延迟,从而给用户带来更好的体验。\par
而在无线网络架构中,存在有多个的利益相关方,包括有提供下载内容的服务提供商、管理无线电接入网设施的移动网络运营商以及希望享受不同内容的移动用户,在使用特定的缓存策略时,不同的利益相关方为了各自的利益可能相互冲突。由于各方只关心自己的利益,因此有必要分析这些各方之间的相互作用,并设计适当的解决方案。基于此博弈论可能是解决这类问题的一个有效工具。\par
从研究方向来看,无线通信网络缓存主要研究缓存设计,利用5G网络的预测能力,结合关于机器学习和博弈论等方向的研究,设计出合理的缓存方案。可以预见,作为5G通信的一个重要研究方向,无线通信网络缓存将在提升用户体验、减少基站流量,降低回程链路负载等方面做出贡献。
\section{研究现状}
近年来移动智能设备的普及极大地丰富了人们在移动互联网上的体验,也带来了包括多媒体业务、网络浏览和社交网络等大量新的无线服务。受限于无线通信传输中有限的频谱资源以及基站与用户间的通信链路或基站与核心网络间的无线回程链路的拥塞问题,为满足这些新的业务需求,一种有效的方法就是将受欢迎的内容缓存在移动网络的前端。这种无线通信网络中的缓存的设计就是本课题所需要研究的问题。\par
对于无线通信网络中的缓存的设计问题,文献\cite{andrews2013seven}提出部署小型蜂窝网络(SCN)是一种可行的方法。小型蜂窝网络是一种新型网络模型,他将短距离、低功率和低成本小型基站部署在宏蜂窝网络下。文献\cite{andrews2013seven}已经处理了与自组织、小区间干扰协调(ICIC)传输卸载、能量效率等有关的问题。这些研究是在反应网络模型下进行的,不能满足峰值流量的需求,其大规模部署也会带来高昂的站点获取、安装和回程成本。文献\cite{golrezaei2012femtocaching}提出了微基站高速缓存(FemtoCaching)的思想,其中基站具有低带宽(可能是无线的)回程链路和高存储能力。文献\cite{tadrousproactive}中探讨了主动资源分配的概念,其利用用户行为的可预测性来进行负载平衡。其中,通过使用来自大偏差理论的工具,中断概率的缩放规律被导出为一个关于预测时间的函数。类似地,文献\cite{ji2013fundamental}研究了设备到设备(D2D)通信中缓存的渐近缩放规律,其中用户会缓存受欢迎的内容并利用D2D通信来交流信息。\par
在文献\cite{bastug2014living}中,讨论了当前反应网络的局限性,提出了利用主动缓存的动态性和5G网络的预测能力,结合关于机器学习和社交网络等领域的研究,通过在基站和用户设备处进行缓存来主动地服务可预测的用户请求,以降低峰值业务需求。在文献\cite{bastug2015proactive}中,又对SCN中的主动缓存的模型进行了研究探索,提出了支持缓存的SCNs,其使用机器学习中的工具,利用用户可预测的行为和他们的社交关系在网络边缘的设备处进行缓存,包括在基站处的缓存和在用户终端处的缓存。\par
文献\cite{bastug2014anticipatory}指出,主动缓存过程被论证为一个有监督机器学习问题,并且协同过滤(CF)技术被用于估计文件受欢迎度矩阵,传统的基于协同过滤的学习方法,分为训练和预测两部分,在训练部分需要初步得出文件受欢迎度矩阵,在预测部分中,使用CF技术估计文件受欢迎度矩阵中的未知元素,得到受欢迎度矩阵与文件和用户矩阵之间的近似关系。\par
文献\cite{bastug2014living}提出了另一种方案,他使用监督机器学习和特殊的CF方法,区别于前一种方案,他利用特殊协同过滤方法,其利用正则化的奇异值分解技术控制其数值精度,将已知的受欢迎程度矩阵构造为需估计的矩阵的低级版本。由于训练集的稀疏性质,可以利用最小二乘性质通过梯度下降来进行分解,然后再进行缓存的安排。文献
\cite{bharath2016learning}考虑了一个具有宏基站、小型基站和以独立泊松点过程分布的用户的异构网络。回程负载通过随机缓存策略的成本函数来捕获。根据成本函数的估计,从而设计最优随机高速缓存策略。如果用户密度大于预定阈值,则所需的训练时间是有限的。因此提出了一种基于学习的方法来改进这种估计。当使用参数分布族对受欢迎文件概况建模时,训练时间减少且延迟随分布参数的维度线性地缩小。\par
然而,在实际中文件受欢迎度矩阵通常很大并且稀疏,使得CF技术在数据稀疏和冷启动问题上的表现欠佳,这给我们基于学习的缓存方案设计带来了机器学习领域的挑战,促使我们去寻求更好的学习方法来提高缓存方案的性能。\par
对于无线通信缓存问题,文献\cite{hu2016game}从博弈论的角度探讨了用于不同无线缓存环境的缓存方案。他指出有效的缓存策略可以通过使用博弈论考虑各方之间的关系和交互来进行设计,包括用博弈论模拟和分析无线主动缓存中各方的合作与非合作行为。其中讨论了几个典型的缓存环境,并应用相应的博弈论模型,说明了各个利益相关方的自私行为会如何影响整体的无线主动缓存。其中在集中式无线网络中,利用拍卖博弈来解决小基站缓存环境下多个博弈者之间的竞争问题,在分布式无线网络中,应用合作博弈分析D2D 缓存环境下的协作缓存的可能性,而进化博弈是分析V2V缓存中用户行为的有效工具,同时针对小蜂窝网络缓存环境,提出了一个基于拍卖博弈的多对象解决方案。\par
文献\cite{tan2016femtocaching}研究了缓存系统中的分布式缓存问题。这个问题是一个NP完全问题,其借助博弈论思想提出了一个分布式框架,通过找到博弈论问题的纳什均衡来解决缓存问题。这些缓存问题中博弈论的应用也给我们基于学习的缓存方案设计带来了启发。\par
目前无线通信网络中的缓存还存在很多的问题需要解决。首先是基于学习的缓存过程实现方案不多,并且传统的方案有明显缺陷。其次,由于移动数据的大幅增长,缓存问题中用户对于文件的需求量很大,且会随时间与空间动态变化,导致利用机器学习技术进行预测十分困难。对于不同网络环境,需要有不同的博弈论方法来指导缓存分配的决策。而目前主动缓存方法仍处于起步阶段,当前主要只能从上层的角度进行研究。
\section{本文的主要内容}
本文首先简单介绍了主动缓存,而如何进行主动缓存正是本文的主要内容,其中包括了主动缓存提出的必要性,以及它的特点和主要思想。分析了主动缓存涉及到的技术领域,和我们实现主动缓存时所可能面临的一些问题。\par
在第三章中,我们具体介绍了利用机器学习领域的方法来解决主动缓存问题的内容。包括使用合适的协同过滤方法,即基于奇异值分解的协同过滤算法,并以此为基础,结合迁移学习的思想对算法进行优化并提出了基于迁移学习的缓存算法。实现了对缓存问题中用户与文件之间关系的估计和预测。通过建立模型和仿真,讨论了我们基于迁移学习的缓存算法的性能。\par
在第四章中,我们从博弈论的角度分析了小基站如何对内容文件进行缓存的问题。通过局部合作博弈思想,设计了一个分布式迭代算法来找到博弈的最好的纳什均衡,最佳纳什均衡就是分布式缓存问题的全局最优解。然后建立模型并仿真,对比验证了该博弈缓存算法的性能优越性。\par
本文的主要贡献如下:
\begin{enumerate}
\item 在解决主动缓存问题时,为了从初始可能极其稀疏的内容文件受欢迎度信息中估计出完整的文件受欢迎度矩阵,我们以传统的协同过滤算法为基础,通过对矩阵奇异值分解技术的目标函数进行不断优化,提出主动缓存问题下经过优化后的基于奇异值分解的协同过滤算法。
\item 为了更好地应对数据的稀疏性等问题,我们结合迁移学习的思想,以待解决的缓存问题为目标域,以用户间D2D通信、社交网络相关的信息交换为源域,通过对源域的大量先验信息的学习来更好地预测目标域中的信息,并提出了基于迁移学习的内容缓存算法。
\item 在进行小基站的缓存决策时我们从博弈论的角度,将分布式缓存问题视为一个严格势博弈问题,并以此为基础设计了博弈缓存算法,来找到严格势博弈的最佳纳什均衡,也即该缓存问题的全局最优解。
\end{enumerate}

\chapter{主动缓存概述}
\section{主动网络的提出背景}
近年来智能手机等移动设备的普及大大丰富了移动用户的体验,也带来了大量新的无线服务,包括多媒体,网络浏览和社交互联网络。这种现象进一步推动了移动视频流的发展,它目前占移动数据流量的近50%,并且预计在未来10年会增加500倍\cite{cisco2014global}。与此同时,第二大流量贡献者就是社交网络,其平均份额为15%\cite{Ericsson20125g}。这些现实情况都促使移动运营商去重新设计他们当前的网络,以寻求更先进和尖端的技术来增加网络的覆盖范围、提高网络容量,并且使内容更有效地贴近用户。\par
为满足这些前所未有的业务需求,部署小蜂窝网络(SCN)是一种可行的方法 \cite{andrews2013seven}。小蜂窝网络是基于现有的反应网络模型,他要求用户的业务请求和数据流必须在其到达或被丢弃时立即得到服务,而这会导致中断的出现。因此,现有的SCN模型不能满足峰值流量需求,其大规模部署也会带来高昂的站点获取、安装和回程成本。随着连接设备数量的激增以及超密集网络的出现,这些缺点将变得越来越严重,且会对当前的蜂窝网络基础设施造成损耗。解决这些问题的关键是构建一种超越当前异构小蜂窝网络的新网络模型,这种网络模型可以利用存储,情境感知和社交网络等领域的发展进行主动缓存。
\section{主动网络概述}
主动网络的 “主动性”植根于所有网络节点(即基站以及手持设备和智能电话),利用用户的情境感知,预测用户的需求并利用其预测能力来实现资源节省,同时保证服务质量要求(Qos)和成本/能源支出\cite{bastug2013proactive}。这种模型优越于目前的蜂窝网络部署方式,因为在蜂窝网络设计时,是假设设备是“傻瓜”的:存储和处理能力非常有限。然而,当前的智能手机已经变成一种非常复杂的设备,它具备了更强的存储能力和计算能力。因此,在主动网络模型下,利用设备的功能和大量的可用数据,可以跟踪、学习网络中的用户并构建他们的需求简档以预测其未来可能的请求。使用机器学习技术来提取和分析大量的基础设施日志,以产生用于预测和内容推荐的、具有预测性和可操作性的信息
\cite{etter2012been}。具备了这些预测能力,通过在非高峰时间(例如夜间)主动提供对高峰时段需求的预测,可以用更有效的方式来调度用户,并且能更智能地预分配资源。这种主动网络模型通过巧妙地利用用户的上下文信息(即,文件受欢迎度的分布,用户位置,速度和移动性模式),能够更好地预测何时请求用户的内容、所需的资源量以及哪些网络位置的内容应该被预先缓存。
\section{主动网络框架下的主动缓存}
主动网络框架是在网络节点预测用户的需求,并利用它们的预测能力来降低网络流量的峰值均值比,这会使网络资源得到显著的节省。这种主动缓存方式利用了现有的异构蜂窝网络,涉及了预测无线电资源管理技术的设计,最大化了未来第五代(5G)网络的效率。\par
具体来讲,主动缓存基于一种概念,即移动用户的信息需求在某种程度上是可预测的。可以利用这种可预测性,通过在用户实际请求它之前,就已经主动地预先缓存所需的信息来最小化蜂窝网络的峰值负载。网络运营商可以利用智能手机的强大处理能力和大量存储空间,在非高峰时间主动提供关于峰值时断请求的预测。也就是说,当主动网络在截止时间前接收到用户的服务请求时,将相应的数据存储在用户设备中,并且当实际发起请求时,可以直接从缓存的存储器中拉出信息,不需要访问无线网络。为此,我们需要开发新的机器学习技术,以找到关于内容检索预测的最佳方式,解决用户从未发出请求和请求不能够及时服务的问题。显然,以本地方式在基站和用户终端分析用户的业务和缓存内容可以减少回程流量,特别是当网络接收到大量内容类似的请求时回程节省会更加显著。因此,我们的目标是以智能方式预测和推断未来事件,我们认为这是由大量并且稀疏的信息/数据引起的复杂大数据问题
\cite{laurila2012mobile}。事实上,数据的稀疏性是一个关键的挑战,因为我们可能不能从单个用户收集足够的数据,因而不能够准确地预测她/他的模式。为了克服这个挑战,可以利用其他用户的数据以及他们的社会关系来建立可靠的统计模型。而我们缓存设计最重要是要解决,要如何建立用户与缓存内容之间的关系,哪些内容应该被优先缓存,用何种方式来实现缓存,应在网络中的哪个位置或是基站进行缓存等问题。
\section{主动缓存的实现方式}
为了实现所提出的主动缓存,我们主要从迁移学习和博弈论两个角度来分析和解决问题。第三章和第四章会分别具体介绍这两个方法。
\subsection{迁移学习}
在主动缓存时,我们需要预测和估计文件与用户之间的相关度,即文件对于用户的受欢迎度矩阵。传统的解决方案是使用协同过滤方法,然而实际上初始的文件受欢迎程度矩阵较大、很稀疏、有大量的未知元素,造成协同过滤算法性能低下,并有数据稀疏、冷启动等问题。显然简单对传统协同过滤进行优化是不够的,而使用迁移学习能较好地解决这些问题。\par
迁移学习的提出,是由于在传统机器学习中构建模型可能需要海量的数据,这些数据受限于实际的情况是无法得到的,而迁移学习可以通过在不同但又相关的两个领域进行知识的迁移来解决这个问题。具体到我们主动缓存的问题中,我们可以从社交网络中用户之间互相通信、进行交互的信息,获取该领域中用户与所请求的内容之间的关系,来帮助我们实现迁移学习,更好地预测和估计主动缓存领域中文件与用户的受欢迎度矩阵。
\subsection{博弈论}
博弈论又称对策论,它是研究多个个体或团队在特定约束条件下的对弈中相互之间利用相关方的策略而采取对应策略的一种理论,是运筹学的一个分支,同时广泛应用于经济学、社会学、计算机科学等领域,现在我们将其应用在分布式缓存问题中。\par
在分布式缓存系统中,通过在小基站缓存内容可以为移动用户提供服务并降低回程负载。我们知道,这个分布式缓存问题是一个NP完全问题。我们可以从博弈论的角度来思考这个问题,将分布式缓存问题用博弈论进行建模,将每一个小基站视为一个博弈者并有缓存大小的限制,博弈者的策略为该小基站对于用户可能请求的内容是否进行缓存,博弈的效用函数就是可以被小基站服务的用户数。\par
而博弈论中有一种特殊的博弈,称它为势博弈,即在博弈当中,所有的个体(博弈者)改变策略而造成的效用的改变都能被映射到全局函数上,此全局函数即称势函数,我们研究势博弈的一个重要意义在其可以帮我们分析博弈的均衡,因为所有势博弈是一定存在有纯纳什均衡(PNE)的。而在势博弈中,严格势博弈又是一种特殊的势博弈。更进一步地说,可以证明我们的分布式缓存问题是一个严格势博弈。因此,我们可以设计算法找到该严格势博弈的最好的纳什均衡,这也就是我们分布式缓存问题的全局最优解。
\section{小结}
本章简要介绍了主动缓存技术的相关内容,包括它的提出背景,以及它的特点、作用和主要需解决的问题。在设计无线通信网络中的分布式缓存方案时我们使用的就是主动缓存技术,通过在网络边缘设备进行缓存来主动地服务可预测的用户需求。下文将具体讨论实现主动缓存的算法设计并提出缓存方案。
\chapter{基于迁移学习的主动缓存方案设计}
\section{问题描述}
在本章中,我们具体讨论无线通信网络中的主动缓存技术。在我们的网络模型中,小基站(SBS)部署有大容量的存储单元,但回程链路的容量有限。文献
\cite{bastug2013proactive}中提出了一种主动缓存过程,它是基于文件的受欢迎程度来存储文件,即缓存最受欢迎的文件,直到将缓存填满,而我们的研究也是基于这种方法。\par
首先我们假设网络中有$N$个用户和$F$个文件,文件的受欢迎程度矩阵为$\textbf{P}_{N\times F}$,矩阵中每行代表用户,而列代表文件,每个元素代表相应用户对相应文件的评分(评级)。在主动缓存过程中,我们需要有矩阵$\textbf{P}_{N\times F}$的完整信息。我们的主动缓存过程由训练和缓存安排两部分组成。在训练部分中,我们的目标就是估计受欢迎程度矩阵$\textbf{P}_{N\times F}$。\par
在实践中,文件的受欢迎程度矩阵较大、很稀疏、有大量的未知元素,因此我们就需要利用这个十分稀疏的矩阵的已知信息来预测矩阵中其他未知的元素。在解决这个问题时,我们受到Netflix推荐系统架构的启发,使用了监督机器学习和协同过滤(CF)的方法,提出了一个分布式主动缓存的过程,利用用户与文件的相关性来推断第$u$个用户请求第$i$个文件的概率。在利用CF技术进行机器学习的具体方法上,我们是通过正则化的奇异值分解(SVD)技术来完成的,即基于SVD的协同过滤。\par
然而,由于初始的受欢迎程度矩阵通常很大,稀疏,用户评级很少,就导致了CF学习方法的效率比较低下,这就是所谓的数据稀疏(data sparseness)和冷启动(cold-start)问题
\cite{lee2012comparative}。鉴于数据稀疏和冷启动问题降低了主动缓存的性能,我们考虑到了结合迁移学习(TL)框架及相关技术来进行机器学习和协同过滤\cite{pan2010survey}。使用TL是因为在许多现实世界的应用中,很难甚至不可能收集和标注足够的训练数据来构建合适的预测模型,但我们如果有其他相关的信息来源的大量数据信息,就可以利用这些丰富的数据来帮助我们来进行学习。具体的,在我们主动缓存问题中,设备与设备间(D2D)相互通信、交互的信息,即社交网络中的用户信息,就可以用来帮助我们实现迁移学习。我们将这种D2D交互的信息域称为源域(Source domain),而我们所要解决的主动缓存问题所在信息域称为目标域(Target domain)。将源域中这些现有信息并入目标域中,可以用来优化我们的主动缓存技术。\par
接下来,我们将构建目标域与源域的网络模型,并具体介绍我们的基于SVD的协同过滤方案,以及结合迁移学习来进行优化后的解决方案。
\section{网络模型}
由于目标域中可能缺乏足够的先验信息,我们从用户基于D2D的社交互动中提取信息,将这些信息通过学习迁移到目标域。我们假设源域中的一个信息系统为$S^{(S)}$,目标域中的一个信息系统为$S^{(T)}$。网络模型的一个示例如图\ref{TLnm}所示。
\begin{figure}{}
\centering
\includegraphics[height=8cm ,width=7cm]{TLnm.jpg}
\caption{网络模型示意图}\label{TLnm}
\end{figure}
\subsection{目标域}
我们考虑一种网络部署方案,它由$M_{tar}$个SBSs和$N_{tar}$个用户终端(UT)构成。每个SBS $m$ 都通过一个回程链路连接到核心网络(CN)上,回程链路容量记为$C_m$,每个SBS 通过无线连接来为UTs提供服务,总无线链路容量记为$C'_m$。UTs请求的内容来源于一个由$F_{tar}$个文件构成的文件库中,其中每个内容$f$的大小为$L(f)$,比特率要求为$B(f)$。 此外,我们认为用户对内容的请求服从Zipf分布\cite{breslau1999web}:
\begin{equation}\label{Zipf}
P_f=\frac{1/f^\alpha}{\sum_{i=1}^{F_{tar}}1/i^\alpha}
\end{equation}
其中$\alpha$是描述分布特性的指数因子,反映了不同的文件受欢迎度分布情况。有了上述的文件受欢迎度分布情况,那么第$m$个SBS在时间$t$时的文件受欢迎度矩阵可以用$\textbf{P}^m(t)\in R^{N_{tar}\times F_{tar}}$来表示,其中每个元素$P_{n,f}^m(t)$表示第$n$个用户请求第$f$个文件内容的概率。\par
为了模拟用户请求内容时进行内容传送的情况,我们假设每一个SBS有固定的存储容量,大小为$S_m$,并从$F_{tar}$个文件中选择内容进行缓存。SBS能够依靠他们的本地缓存来对用户请求进行服务,对于减少峰值的请求量以及最小化内容传送的延迟是相当重要的。我们通过在合适的时间提前将CN中的内容,根据SBSs的存储容量情况来缓存在其中,来达到满足用户对内容的请求的同时,减少回程负载。为了实现这种想法,假设在时隙$T$内,用户发出的请求数量为$D$,请求集合记为$\mathcal{D}=\{1,\dots,D\}$。如果一个请求$d\in\mathcal{D}$ 在时间窗$T$内能被立即服务,那么我们认为该请求被满足,即内容传送的速度大于等于内容的比特率。因此用户的平均满意率可以由下式表示:
\begin{equation}
\eta(\mathcal{D})=\frac{1}{D}\sum_{d\in\mathcal{D}}L\{\frac{L(f_d)}{\tau'(d)-\tau(d)}\geq B(f_d)\}
\end{equation}
其中,$f_d$是所请求的内容,$L(f_d)$和$B(f_d)$分别是该内容的大小和比特率,$\tau(d)$和$\tau'(d)$分别是请求的到达时间和传送结束时间。$L\{\dots\}$是布尔函数,如果其中的表达式满足返回1,否则是0。假设在$t$时刻请求$d$的内容传送的瞬时回程速率为$R_d(t),R_d(t)\leq C_m$。 那么平均回程负载可表示为:
\begin{equation}
\rho(\mathcal{D})=\frac{1}{D}\sum_{d\in\mathcal{D}}\frac{1}{L(f_d)}\sum_{t=\tau(f_d)}^{\tau'(f_d)}R_d(t)
\end{equation}
现在,记$\textbf{X}(t)\in\{0,1\}^{M_{tar}\times F_{tar}}$为SBSs的缓存决策矩阵,其中$x_{m,f}(t)$等于1表示在时间$t$第$f$个内容被第$m$个SBS缓存,等于0表示未缓存。为了简化问题,我们将SBSs的下标去掉,并假设在时隙$T$内内容的受欢迎度是一致的,因此$\textbf{P}^m(t)$ 可记为$\textbf{P}_{tar}$。我们将缓存策略限制为在非高峰时段内缓存文件,因此在内容传送时$\textbf{X}(t)$是固定的,记为$\textbf{X}$。这样我们目标域的主动缓存问题就归结为预测内容受欢迎度矩阵$\textbf{P}_{tar}$,从而得到缓存策略矩阵$\textbf{X}$。\\
接下来,由于源域的信息可以用来帮助我们解决目标域中$\textbf{P}_{tar}$的稀疏性问题,下面我们将考虑源域的情况。
\subsection{源域}
受到\cite{bastug2014living}的启发,我们可以利用一个基于D2D的社交网络,该社交网络由社交社区内的用户互动形成,我们将此信息域称为源域。这个源域包含用户在所在的社区内的交互行为,这种社交行为可以用中国餐馆过程(CRP)\cite{griffiths2011indian}来进行模拟。它们构成了迁移学习所需要用到的先验信息。\par
我们首先将源域中的用户数量定义为$N_{D2D}$,将内容总数定义为$F_{D2D}$。我们假设$F_{D2D}=F_0+F_h$,其中$F_h$表示有历史记录的内容数量,$F_0$表示没有历史记录的内容数量。在社交网络中,用户利用他们的社会关系寻求他们各自所需的内容。我们假设每个用户在所提供的内容中仅对一种类型的内容$f$感兴趣。我们假定内容(文件)$f$被给定用户$n$选择的概率遵循Beta分布(即先验分布)。因此,用户$n$的选择结果被定义为Beta分布(先验分布)的共轭概率,其遵循伯努利分布。考虑到这一点,用户选择文件的过程可模拟为CRP。CRP是基于一种比喻的说法,其中对象是餐馆中的客人,而类是他们所坐的桌子。特别地,在有着大量桌子的餐馆中,每个桌子具有无数个座位,顾客一个接一个地进入餐馆,并且每个人都随机选择桌子。在具有参数$\beta$的CRP中,每个客人以一个与占用者数量成比例的概率选择一张已被占用的桌子,而以与$\beta$成比例的概率选择下一空桌。具体的,第一个客人以$\beta/\beta=1$的概率选择第一张桌。第二个客人以$1/(1+\beta)$的概率选择第一张桌,以$\beta/(1+\beta)$的概率选择第二张桌。在第二个客人选择第二张桌之后,第三个客人以$1/(2+\beta)$的概率选择第一张桌,$1/(2+\beta)$的概率选择第二张桌,$\beta/(2+\beta)$的概率选择第三张桌。此过程一直持续到所有客人都有座位,该过程从而定义了客人与桌子间的分配。因此可以看出,后续客人的决策受到先前客人的反馈的影响,其中每个客人从先前客人的选择中学习以更新他们对桌子的看法和选择桌子的概率。\par
鉴于此,社交网络中的内容传播类似于CRP中的桌子选择。事实上,如果我们将所研究的网络视为中国餐馆,内容文件视为非常大量的桌子,而用户视为客人,那么我们可以通过CRP来理解内容的传播过程。也就是说,在每个社交圈内,用户按顺序请求下载他们所需的内容,并且当用户下载其内容时,进行记录(即下载历史)。反过来,该行为影响了社交社区内的其他用户请求这个相同内容的概率,其中受欢迎的内容会被更频繁地请求而新内容被请求的较少。记$\textbf{Z}_{D2D}\in\{0,1\}^{N_{D2D}\times F_{D2D}}$为指示每个用户选择哪些内容的随机二进制矩阵,其中如果用户$n$选择内容$f$,则$z_{n,f}=1$,否则为0。该矩阵概率可以表示如下\cite{griffiths2011indian}:
\begin{equation}
P(\textbf{Z}_{D2D})=\frac{\beta^{F_h}\Gamma(\beta)}{\Gamma(\beta+N_{D2D})}\prod_{f=1}^{F_h}(m_f-1)!
\end{equation}
其中$\Gamma(\cdot)$是伽马函数,$m_f$是选择了内容$f$的用户数(即访问历史),$F_h$是有访问历史(即$m_f>0$)的内容的数量。\par
在目标域中,缓存问题归结为估计内容受欢迎度矩阵,而我们认为该矩阵十分的稀疏、大量元素是未知的,这导致了主动缓存性能较低。而在用户数量和内容文件数量越来越大的的现实情况下,这种问题可能会更加严重。因此,为了更有效地处理这些问题并缓存内容文件,我们提出了一种新的结合迁移学习的主动缓存技术,其中利用了从用户的社交网络交互中提取的丰富的信息(即源域信息)。
\section{基于奇异值分解的协同过滤算法}
协同过滤方法由训练和预测两个步骤构成。在训练阶段,我们的目标就是估计内容受欢迎度矩阵$\textbf{P}_{tar}\in R^{N_{tar}\times F_{tar}}$。其中每个SBS基于已知的信息(即用户对内容的评价)构建模型。在目标域中,稀疏的内容受欢迎度矩阵$\textbf{P}_{tar}$中的元素为$P_{tar,ij}$。在利用协同过滤估计内容受欢迎度矩阵时,我们将矩阵变换为如下的等价形式:$\textbf{R}_{tar}=\{(i,j,r):r=P_{tar,ij},P_{tar,ij}\neq0\}$,每行三个元素分别代表用户id $i$、内容文件id $j$和用户对文件的评分$r$,该矩阵记录了所有已知的用户评价信息。为了预测$\textbf{P}_{tar}$中的未知元素,我们利用SVD对矩阵进行分解,构造一个秩为$k$的受欢迎度矩阵的一个估计$\textbf{P}_{tar}\approx \textbf{N}_{tar}^T\textbf{F}_{tar}$,其中因子矩阵分别为$\textbf{N}_{tar}\in R^{k\times N_{tar}}$和$\textbf{F}_{tar}\in R^{k\times F_{tar}}$,这两个因子矩阵的物理意义可以理解为,每个用户和文件分别有$k$个特征,矩阵的每行就是对应用户或文件的特征向量。那么显然该优化问题的目标函数可以写为预测值和真实值的误差平方和的形式:
\begin{equation}
\min_{i,j\in\textbf{P}_{tar}}\sum_{(i,j)\in\textbf{P}_{tar}}(P_{tar,ij}-n_i^Tf_j)^2
\end{equation}
其中训练集中的用户-文件对记为$(i,j)$,$n_i$和$f_j$分别表示$\textbf{N}_{tar}$和$\textbf{F}_{tar}$的第$i$和第$j$列。然而,若按照该公式针对已知评分数据进行训练,会过分地拟合这部分数据,这会导致模型的训练效果很差,这就是过拟合问题。为了避免过拟合的问题,我们在目标函数中加入正则化参数,该参数用来在正则化和拟合训练数据这两者之间进行平衡,因为对于目标函数来说,$n_i$矩阵和$f_j$矩阵中所有的值都是变量,而我们在不知道哪个变量会造成过拟合问题的情况下,对其中所有变量进行“惩罚”,这就是正则化的SVD,即RSVD,此时目标函数变为如下形式:
\begin{equation}
\min_{i,j\in\textbf{P}_{tar}}\sum_{(i,j)\in\textbf{P}_{tar}}(P_{tar,ij}-n_i^Tf_j)^2+\lambda(|n_i|^2+|f_j|^2)
\end{equation}
下面,我们对上述式子进行优化。为了更好地描述用户-文件对,我们将用户-文件对记为$(u.i)$,我们算法原始受欢迎度矩阵(即输入的训练矩阵)是$\textbf{R}_{tar}$,因此$P_{tar,ui}$ 对应于$\textbf{R}_{tar}$ 中的$r_{ui}$ 就是已知的评分真实值,将对应的特征向量$n_i$和$f_j$分别记为$p_u$ 和$q_i$,那么预测值就是$\hat{r}_{ui}=p_u^Tq_i$,为了方便,我们记真实值与预测值的误差为$e_{ui}=r_{ui}-\hat{r}_{ui}$。由于用户对文件的评分不仅取决于用户和文件之间的某种关系,还取决于用户和文件各自特有的性质,因此我们对预测值的公式加入基准线预测器,即$\hat{r}_{ui}=\mu+b_u+b_i+p_u^Tq_i$,其中$\mu$为总平均评分,$b_u$ 为用户$u$ 的属性相对于平均值$\mu$的偏移,$b_i$为文件$i$ 的属性相对于平均值$\mu$ 的偏移,$b_u$ 和$b_i$同样需要有正则化参数进行平衡,综上目标函数变为如下形式:
\begin{equation}
\min_{u,i\in\textbf{P}_{tar}}\sum_{(u,i)\in\textbf{P}_{tar}}(r_{ui}-\mu-b_u-b_i-p_u^Tq_i)^2+\lambda(b_u^2+b_i^2+|p_u|^2+|q_i|^2)
\end{equation}
有了上述的目标函数,我们只需要通过训练对它进行优化,我们的优化算法为随机梯度下降法。具体的,目标函数分别对变量$b_u$、$b_i$、$p_u$和$q_i$求偏导,再将这四个变量向负梯度方向变化,由此可得到四个变量的更新式如下:
\begin{equation}
b_u=b_u+\gamma(e_{ui}-\lambda b_u)
\end{equation}
\begin{equation}
b_i=b_i+\gamma(e_{ui}-\lambda b_i)
\end{equation}
\begin{equation}
p_u=p_u+\gamma(e_{ui}q_i-\lambda p_u)
\end{equation}
\begin{equation}
q_i=q_i+\gamma(e_{ui}p_u-\lambda q_i)
\end{equation}
在通过上述的CF算法进行训练时,我们需要了解当前估计值与真实值之间的偏差,对于误差的度量我们用“均方根误差”来度量。我们假设测试值矩阵为$\textbf{R}_{test}$,该矩阵与$\textbf{R}_{tar}$ 类似,每行3个数分别代表用户id、内容文件id和相应评分,显然测试值个数为$\textbf{R}_{test}$的行数$T_E$,现在我们将评估指标给出:
\begin{equation}
RMSE=\sqrt{\sum_{(u,i,r_{ui})\in R_{tar}}(r_{ui}-\hat{r}_{ui})^2/T_E}
\end{equation}
下面我们简要的给出基于SVD的CF算法的主要伪代码。
\begin{algorithm}
    \caption{基于SVD的CF(1)}
    \begin{algorithmic}[1] %每行显示行号
        \Require $R_{tar}$原始训练矩阵,$R_{test}$测试矩阵
        \Ensure 估计矩阵$P$,评估指标$RMSE$
        \Function {LoadFileAndInitial}{$R_{tar},R_{test}$}
            \State Load $R_{tar},R_{test}$
            \State create arrays $bu[UserNum+1],bi[ItemNum+1],p[UserNum+1,dim]$
            \State create arrays $q[ItemNum+1,dim],RateMatrix[UserNum+1][ItemNum+1]$
            \State $RateMatrix,mean \gets R_{tar}$
            \For{every element in $bu,bi,p,q$}
                \State $bu\gets0,bi\gets0,p\gets rand()/10,q\gets rand()/10$
            \EndFor
        \EndFunction
        \Function {Train}{$gamma,lambda,nIter$}
        \State $Rmse\gets0,LastRmse\gets100000,RateNum\gets0,rui\gets0$
        \For{$n=1\to nIter$}
            \State $Rmse\gets0,RateNum\gets0$
            \For{$i=1\to UserNum$}
                \For{$j=1\to IterNum$}
                    \State $rui\gets mean+bu[i]+bi[j]+p[i]\cdot q[j],e\gets RateMatrix[i][j]-rui$
                    \State $bu[i]\gets bu[i]+gamma*(e-lambda*bu[i])$
                    \State $bi[j]\gets bi[j]+gamma*(e-lambda*bi[j])$
                    \For{$k=0\to dim-1$}
                        \State $p[i][k]\gets p[i][k]+gamma*(e*q[j][k]-lambda*p[i][k])$
                        \State $q[j][k]\gets q[j][k]+gamma*(e*p[i][k]-lambda*q[j][k])$
                    \EndFor
                    \State $Rmse\gets Rmse+e^2,RateNum\gets RateNum+1$
                \EndFor
            \EndFor
            \State $Rmse\gets\sqrt{Rmse/RateNum}$
            \If{$Rmse>LastRmse$}
                \State $break$
            \EndIf
            \State $LastRmse\gets Rmse,gamma\gets gamma*0.9$
        \EndFor
        \For{$i=1\to UserNum$}
                \For{$j=1\to IterNum$}
                    \State $P[i][j]\gets mean+bu[i]+bi[j]+p[i]\cdot q[j]$
                \EndFor
        \EndFor
        \State \Return{$P$}
        \EndFunction
    \end{algorithmic}
\end{algorithm}
\newpage
\begin{algorithm}
    \caption{基于SVD的CF(2)}
    \begin{algorithmic}[1] %每行显示行号
        \Function {Predict}{}
        \State $Rmse\gets0,Num\gets0$
        \For{every row $i$ of $R_{test}$}
            \State $userid\gets R_{test}[i][0],itemid\gets R_{test}[i][1],rate\gets R_{test}[i][2]$
            \State $rui\gets mean+bu[userId]+bi[itemId]+p[userId]\cdot q[itemId]$
            \State $Rmse\gets Rmse+(rate-rui)^2,Num\gets Num+1$
        \EndFor
        \State $Rmse\gets\sqrt{Rmse/Num}$
        \State \Return{$Rmse$}
        \EndFunction
        \Function {main}{}
        \State \Call{LoadFileAndInitial}{$R_{tar},R_{test}$}
        \State \Call{Train}{$gamma,lambda,nIter$},\Call{Predict}{}
        \EndFunction
    \end{algorithmic}
\end{algorithm}
\section{结合迁移学习的内容缓存算法}
为了应对目标域中数据稀疏的问题,我们利用来自与目标域不同但有一定相关性的源域的大量可用信息(即先验信息),进行迁移学习,可以更有效地解决主动缓存问题。我们记源域为$S^{(S)}$,并假设源域中有$N_{D2D}$个用户和$F_{D2D}$个内容文件,他们的集合分别为$\mathcal{N}_{D2D}$和$\mathcal{F}_{D2D}$,源域中的用户内容受欢迎度矩阵由矩阵$\textbf{P}_{D2D}\in R^{N_{D2D}\times F_{D2D}}$给出,相应的$\textbf{R}_{D2D}=\{(i,j,r):r=P_{D2D,ij},P_{D2D,ij}\neq0\}$表示源域中可知的用户对内容的评价。记目标域为$S^{(T)}$,我们这里采取迁移学习方法的基本思想就是是巧妙地“借用”$S^{(S)}$中用户的社会行为信息,来更好地学习$S^{(T)}$。\par
\begin{figure}{}
\centering
\includegraphics[height=7cm ,width=7cm]{TLcc.jpg}
\caption{基于迁移学习的缓存过程}\label{TLcc}
\end{figure}
基于迁移学习的缓存过程如图\ref{TLcc}所示。从源域到目标域的迁移学习过程分为两个阶段。首先,我们建立源域与目标域之间用户-内容文件的相关性。然后,为了实现知识迁移,结合源域与目标域来建立优化问题,通过学习不断优化目标域中受欢迎度矩阵$\textbf{P}_{tar}$的估计。具体来说,我们假设源域和目标域都与一个信息系统$s\in\{S^{(S)},S^{(T)}\}$ 相关,该信息系统中有$N_{s}$ 个用户和$F_{s}$ 个内容文件,他们的集合分别为$\mathcal{N}_{s}$和$\mathcal{F}_{s}$,令$\textbf{R}_{s}=\{(i,j,r):r=P_{s,ij},P_{s,ij}\neq0\}$表示$s$ 中可知的用户对内容的评价。\par
在利用迁移学习时,我们同样需要建立用户因子矩阵$\textbf{N}\in R^{k\times N_s}$和内容文件因子矩阵$\textbf{F}\in R^{k\times F_s}$,其中他们的第$i$列和第$j$列分别表示为$n_i$和$f_j$,那么相应的我们可以构造受欢迎度矩阵的一个估计为$\textbf{P}_s\approx \textbf{N}_s^T\textbf{F}_s$,与基于SVD的CF算法类似,该问题可以通过优化如下目标函数来解决:
\begin{equation}
\min_{i,j\in\textbf{P}_s}\sum_s\alpha_s\sum_{(i,j)\in\textbf{P}_s}(P_s-n_i^Tf_j)^2+\lambda(|n_i|^2+|f_j|^2)
\end{equation}
其中$\alpha_s$为该信息系统的权重。通过优化该目标函数,矩阵$\textbf{P}_{tar}$和$\textbf{P}_{D2D}$共同被分解,而由于知识共享带来的特征相似性,源域和目标域对应的因子矩阵就会是相互依赖的。下面为了更好地设计算法,我们同样对上述目标函数进行优化,具体方法和步骤与之前的基于SVD的CF算法相似,将用户- 文件对记为$(u.i)$,算法原始受欢迎度矩阵(即输入的训练矩阵)是$\textbf{R}_{s}$,那么$P_{s,ui}$对应于$\textbf{R}_{s}$中的$r_{ui}$就是已知的评分真实值,将对应的特征向量$n_i$和$f_j$分别记为$p_u$ 和$q_i$,预测值就是$\hat{r}_{ui}=p_u^Tq_i$,记真实值与预测值的误差为$e_{ui}=r_{ui}-\hat{r}_{ui}$。同样此时我们要对预测值的公式加入基准线预测器,即$\hat{r}_{ui}=\mu+b_u+b_i+p_u^Tq_i$,其中$\mu$ 为总平均评分,$b_u$ 为用户$u$ 的属性相对于平均值$\mu$ 的偏移,$b_i$为文件$i$的属性相对于平均值$\mu$ 的偏移,最后加入正则化参数进行平衡,综上目标函数变为如下形式:
\begin{equation}
\min_{u,i\in\textbf{P}_s}\sum_s\alpha_s\sum_{(u,i)\in\textbf{P}_s}(r_{ui}-\mu-b_u-b_i-p_u^Tq_i)^2+\lambda(b_u^2+b_i^2+|p_u|^2+|q_i|^2)
\end{equation}
有了上述的目标函数,我们要通过训练对它进行优化,同样采用随机梯度下降法。目标函数分别对变量$b_u$、$b_i$、$p_u$和$q_i$求偏导,再将这四个变量向负梯度方向变化,由此可得到四个变量的更新式如下:
\begin{equation}
b_u=b_u+\alpha_s\gamma(e_{ui}-\lambda b_u)
\end{equation}
\begin{equation}
b_i=b_i+\alpha_s\gamma(e_{ui}-\lambda b_i)
\end{equation}
\begin{equation}
p_u=p_u+\alpha_s\gamma(e_{ui}q_i-\lambda p_u)
\end{equation}
\begin{equation}
q_i=q_i+\alpha_s\gamma(e_{ui}p_u-\lambda q_i)
\end{equation}
估计值与真实值之间的偏差的评估指标与基于SVD的CF算法一样采用“均方根误差”$RMSE$来度量,下面是基于迁移学习的内容缓存算法的主要伪代码
\begin{algorithm}
    \caption{基于迁移学习的内容缓存算法(1)}
    \begin{algorithmic}[1] %每行显示行号
        \Require $R_{tar}$原始训练矩阵,$R_{D2D}$源域信息矩阵,$R_{test}$测试矩阵
        \Ensure 估计矩阵$P$,评估指标$RMSE$
        \Function {LoadFileAndInitial}{$R_{tar},R_{D2D},R_{test}$}
            \State Load $R_{tar},R_{D2D},R_{test}$
            \State create arrays $bu[UserNum+1],bi[ItemNum+1],p[UserNum+1,dim]$
            \State create arrays $q[ItemNum+1,dim],RateMatrix[UserNum+1][ItemNum+1]$
            \State $RateMatrix,mean \gets R_{tar},R_{D2D}$
            \For{every element in $bu,bi,p,q$}
                \State $bu\gets0,bi\gets0,p\gets rand()/10,q\gets rand()/10$
            \EndFor
        \EndFunction
        \Function {Train}{$gamma,lambda,nIter$}
        \State $Rmse\gets0,LastRmse\gets100000,RateNum\gets0,rui\gets0$
        \For{$n=1\to nIter$}
            \State $Rmse\gets0,RateNum\gets0$
            \For{$i=1\to UserNum$}
                \For{$j=1\to IterNum$}
                    \State $rui\gets mean+bu[i]+bi[j]+p[i]\cdot q[j],e\gets RateMatrix[i][j]-rui$
                    \State $bu[i]\gets bu[i]+gamma*alphas(e-lambda*bu[i])$
                    \State $bi[j]\gets bi[j]+gamma*alphas(e-lambda*bi[j])$
                    \For{$k=0\to dim-1$}
                        \State $p[i][k]\gets p[i][k]+gamma*alphas(e*q[j][k]-lambda*p[i][k])$
                        \State $q[j][k]\gets q[j][k]+gamma*alphas(e*p[i][k]-lambda*q[j][k])$
                    \EndFor
                    \State $Rmse\gets Rmse+e^2,RateNum\gets RateNum+1$
                \EndFor
            \EndFor
            \State $Rmse\gets\sqrt{Rmse/RateNum}$
            \If{$Rmse>LastRmse$}
                \State $break$
            \EndIf
            \State $LastRmse\gets Rmse,gamma\gets gamma*0.9$
        \EndFor
        \For{$i=1\to UserNum$}
                \For{$j=1\to IterNum$}
                    \State $P[i][j]\gets mean+bu[i]+bi[j]+p[i]\cdot q[j]$
                \EndFor
        \EndFor
        \State \Return{$P$}
        \EndFunction
    \end{algorithmic}
\end{algorithm}
\newpage
\begin{algorithm}
    \caption{基于迁移学习的内容缓存算法(2)}
    \begin{algorithmic}[1] %每行显示行号
        \Function {Predict}{}
        \State $Rmse\gets0,Num\gets0$
        \For{every row $i$ of $R_{test}$}
            \State $userid\gets R_{test}[i][0],itemid\gets R_{test}[i][1],rate\gets R_{test}[i][2]$
            \State $rui\gets mean+bu[userId]+bi[itemId]+p[userId]\cdot q[itemId]$
            \State $Rmse\gets Rmse+(rate-rui)^2,Num\gets Num+1$
        \EndFor
        \State $Rmse\gets\sqrt{Rmse/Num}$
        \State \Return{$Rmse$}
        \EndFunction
        \Function {main}{}
        \State \Call{LoadFileAndInitial}{$R_{tar},R_{D2D},R_{test}$}
        \State \Call{Train}{$gamma,lambda,nIter$},\Call{Predict}{}
        \EndFunction
    \end{algorithmic}
\end{algorithm}
\section{算法仿真与性能分析}
\subsection{仿真设置}
在本节中,我们将仿真场景设置为一个无线小蜂窝网络。在这个网络模型中,小基站部署有大容量的存储单元,但回程链路的容量有限。我们在仿真时设置了$M_{tar}$个基站和$N_{tar}$个用户。为了简化,假定基站和用户的链路连接以及存储容量都是相同的。用户对于内容文件的请求从大小为$F_{tar}$的文件库中提取,其中的每个文件的大小为$L$,比特率要求为$B$,如果用户的请求能够及时地被基站服务,那么我们就认为他们获得了满意的用户体验。具体的仿真参数如表\ref{RMSEtab}所示。\par
\begin{table}[h]
\centering
\begin{tabular}{*{3}{p{.33\textwidth}}}
\hline \hline
参数 & 描述 & 仿真值\\
\hline $M_{tar}$ & 基站数量 & 4 \\
       $N_{tar}$ & 用户数量 & 128 \\
       $F_{tar}$ & 内容文件数量 & 128 \\
       $\alpha$ & Zipf参数 &2 \\
       $\beta$ & CRP参数 &2 \\
       $\sum S_m$ &存储容量 &10 \\
       $L$ &每个文件的大小&1Mb \\
       $B$ &每个文件的比特率&1Mb/s \\
\hline\hline
\end{tabular}
\caption{仿真参数表}\label{RMSEtab}
\end{table}
在进行算法仿真之前,我们先要获得初始的源域信息矩阵与目标域训练矩阵。对于这两个矩阵,我们先通过源域与目标域中各自的用户- 文件受欢迎度的分布情况给出完整的矩阵信息。而由于在实际情况下矩阵中大量信息是未知的、可知信息是相当稀疏的,因此我们分别在其中随机提取少量的元素作为已知信息,从而得到源域信息矩阵与目标域训练矩阵。具体过程如下:\par
对于目标域,由Zipf分布(\ref{Zipf})随机生成完整的用户-文件受欢迎度矩阵。对于源域,我们通过CRP过程实现用户-文件受欢迎度矩阵的生成。具体方法如算法\ref{CRPalg}所示。\par
\begin{algorithm}
    \caption{模拟CRP生成源域信息}\label{CRPalg}
    \begin{algorithmic}[1] %每行显示行号
    \Ensure 完整的源域信息矩阵$\textbf{P}_{D2D}$
        \State create array $\textbf{P}_{D2D}[UserNum][ItemNum]\gets0,F_{his}[ItemNum]\gets0$
        \For{$i=1\to UserNum$}
            \State $p\gets rand(),k\gets0$
            \For{$j=1\to ItemNum$}
                \If{$F_{his}(j)==0$}
                    \State $\textbf{P}_{D2D}(i,j)\gets\textbf{P}_{D2D}(i,j)+1$
                    \State $F_{his}(j)\gets F_{his}(j)+1$
                    \State $break$
                \EndIf
                \State $k\gets k+F_{his}(j)$
                \If{$p<\frac{k}{beta+i-1}$}
                    \State $\textbf{P}_{D2D}(i,j)\gets\textbf{P}_{D2D}(i,j)+1$
                    \State $F_{his}(j)\gets F_{his}(j)+1$
                    \State $break$
                \EndIf
            \EndFor
        \EndFor
        \State $temp\gets\textbf{P}_{D2D}$
        \State $t\gets randperm(ItemNum)$
        \For{$s=1\to ItemNum$}
            \State $\textbf{P}_{D2D}(:,s)\gets temp(:,t(s))$
        \EndFor
    \end{algorithmic}
\end{algorithm}
在分别生成源域与目标域的矩阵时,我们做了$1000$次Monte-Carlo并求取平均值。在获得完整的源域与目标域的受欢迎度矩阵后,我们在从目标域的受欢迎度矩阵中随机提取5\%的元素作为所有算法的初始训练矩阵$\textbf{R}_{tar}$,从源域中提取10\%的元素作为迁移学习算法的源域信息矩阵$\textbf{R}_{D2D}$。完整的目标域的受欢迎度矩阵可作为对应的测试矩阵$\textbf{R}_{test}$来检测不同的预测估计算法的准确度。\par
接下来,我们分别使用上述的基于SVD的CF算法和基于迁移学习的内容缓存算法,由目标域初始训练矩阵$\textbf{R}_{tar}$和源域信息矩阵$\textbf{R}_{D2D}$得到对于完整的文件受欢迎度矩阵的预测和估计。由于上述的基于SVD 的CF 算法和基于迁移学习的内容缓存算法功能都是实现由稀疏的$\textbf{R}_{tar}$ 到所有真实值的预测。那么在讨论这两个算法时,我们就必须比较两个算法的输出矩阵,即算法得出的用户-内容受欢迎度矩阵的估计值。我们通过仿真,得到了两个算法最终评估指标$RMSE$的对比如图\ref{RMSEp}所示:\par
\begin{figure}{}
\centering
\includegraphics[height=9cm ,width=12cm]{RMSE.jpg}
\caption{RMSE随用户-文件规模变化图(如规模128表示用户与文件数均为128)}\label{RMSEp}
\end{figure}
$RMSE$反映了基于SVD的CF算法和基于迁移学习的内容缓存算法所得出的估计矩阵与真实值的偏差。由图\ref{RMSEp}可知,随着用户规模的逐渐变大,算法的$RMSE$都会越来越小,即算法所得出的估计矩阵越来越接近真实值。图中我们也可以很明显地看出,基于迁移学习的内容缓存算法得到的估计值要更接近于真实值,即基于迁移学习的内容缓存算法具有更好的性能。因此我们在主动缓存时更倾向于使用基于迁移学习的内容缓存算法。\par
下面我们将提出的算法与真实数据缓存(Ground Truth)以及随机缓存(Random caching)进行比较,比较列表如下:
\begin{enumerate}
\item 真实数据缓存(Ground Truth):给定完整受欢迎度矩阵$\textbf{P}_{tar}$并缓存最受欢迎的内容
\item 随机缓存(Random caching):内容被均匀随机缓存
\item 基于迁移学习的缓存(Transfer Learning):使用完整$\textbf{P}_{tar}$5\%的数据和完整$\textbf{P}_{D2D}$10\%的数据进行迁移学习,然后缓存最受欢迎的内容
\end{enumerate}
为了体现各缓存算法的性能,我们假设每个基站的存储容量、内容文件大小、无线链路和回程链路的容量都是一个常数值。我们通过仿真获得了用户满意度的数值结果,具体的仿真参数如表\ref{RMSEtab}所示。下面我们给出了用户满意度随缓存容量大小、小基站数以及源域分布参数变化的仿真性能图。
\subsection{结果分析}
\begin{figure}{}
\centering
\includegraphics[height=9cm ,width=12cm]{TLcaching1.jpg}
\caption{用户满意度随缓存容量变化图}\label{TLp1}
\end{figure}
\begin{figure}{}
\centering
\includegraphics[height=9cm ,width=12cm]{TLcaching2.jpg}
\caption{用户满意度随基站数变化图}\label{TLp2}
\end{figure}
\begin{figure}{}
\centering
\includegraphics[height=9cm ,width=12cm]{TLcaching3.jpg}
\caption{用户满意度随CRP参数变化图}\label{TLp3}
\end{figure}
下面我们详细讨论各算法的性能和各参数的影响。
\begin{enumerate}
\item 缓存容量($S_m$)的影响如图\ref{TLp1}:存储容量大小是基站主动缓存的一个关键参数,显然更大的存储容量会在用户满足度上具有更好的性能。从图中我们可以看到,最好的用户满意度都是真实数据缓存即已知完整的受欢迎度矩阵实现的,而另一方面,随机缓存性能是最差的。而基于迁移学习的缓存算法的性能远高于随机缓存,已经接近于真实数据缓存。
    随着缓存容量的增加,基于迁移学习的缓存算法性能越来越优于随机缓存。
\item 基站数量的影响如图\ref{TLp2}:基站的数量直接影响了主动缓存的性能,显然基站数量越多,用户越容易从基站获取内容文件,用户满足度越高。几个算法中基于迁移学习的缓存算法要明显优于随机缓存,接近与真实数据缓存。而基站的数量越多,我们基于迁移学习的缓存算法的性能优越性越明显。
\item 源域中CRP参数($\beta$)的影响如图\ref{TLp3}:源域中CRP参数($\beta$)在我们的主动缓存问题中也是一个重要的参数,事实上,随着$\beta$的增加,我们基于迁移学习的缓存算法算法的性能逐渐降低,此时需要更高的存储容量来提升性能。同时,基于迁移学习的缓存算法的用户满意度的性能在所有情况下要高于随机缓存,在合理的CRP 参数($\beta$)时接近于真实数据缓存。
\end{enumerate}
\section{小结}
本章具体讨论了无线通信网络中的主动缓存技术,主动缓存过程是基于文件的受欢迎程度来存储文件,即缓存最受欢迎的文件。先以传统的协同过滤算法为基础,通过对矩阵奇异
值分解技术的目标函数进行不断优化,提出主动缓存问题下经过优化后的基于奇异值分解的协同过滤算法。然后为了更好地应对数据的稀疏性等问题,我们结合迁移学习的思想,以待解决的缓存问题为目标域,目标域中的缓存问题归结为估计内容受欢迎度矩阵;以用户间D2D 通信、社交网络相关的信息交换为源域,通过对源域的大量先验信息的学习来更好地预测目标域中的信息,并提出了基于迁移学习的内容缓存算法。最后对算法进行了仿真和分析,很明显对于内容文件受欢迎度矩阵的预测和估计使用迁移学习的方法要比传统的基于SVD的协同过滤方法要准确。具体通过对于缓存性能的仿真,我们可以看出基于迁移学习的主动缓存算法在不同的条件下都要明显优于随机缓存,已经很接近真实数据缓存,证明了我们所提出算法的有效性和优越性。
\chapter{主动缓存中小基站的分布式缓存算法设计}
上一章已经讨论过几种主动缓存的算法。这些算法的主要是通过机器学习等方法,估计、预测出用户- 文件的受欢迎度矩阵,然后贪婪地缓存最受欢迎的文件。而下面我们将从势博弈的角度,利用博弈论的一些方法对于小基站缓存文件的方式进行讨论并提出优化算法。\par
本章我们讨论的是内容文件应存储在哪些基站中,而这个分布式缓存问题,已被证明是一个NP完全问题\cite{golrezaei2012femtocaching}。我们的想法是构建一个分布式的框架,结合博弈论来解决这个问题。在进行主动缓存时使用势博弈方法,当博弈达到最佳纳什均衡(NE)时,就可以获得该分布式缓存问题的全局最优解。具体方法为,先构建系统模型,描述文件,小基站和用户之间的关系。然后证明我们的分布式缓存是一个严格势博弈(EPG)问题,其中每一个小基站作为一个博弈者,他们有着存储容量大小的限制。定义每个博弈者(即小基站)的效用函数为预计基站可以服务的用户数量,那么博弈的每个纳什均衡都是这个分布式缓存问题的局部或全局最优解。接着设计一个分布式缓存算法来找到最优纳什均衡。最后我们对所提出算法进行性能分析。
\section{模型建立}
我们考虑一个移动蜂窝网络,其中有一个宏基站和$N$个小基站,小基站的集合记作$\mathcal{N}=\{1,2,3,\dots,N\}$,有$M$个用户,用户的集合记作$\mathcal{M}=\{1,2,3,\dots,M\}$,每个用户从服务提供商的服务器请求内容文件,文件的集合记作$\mathcal{F}=\{1,2,3,\dots,F\}$。与上一章一样,视频文件的受欢迎程度可以由Zipf 分布(\ref{Zipf})描述。如果用户请求的文件已经被他附近的小基站缓存了,则由该基站对请求提供服务,否则由宏基站对用户提供服务。为了直观展示小基站,用户,内容文件之间的关系,我们可以通过关系图来描述。我们定义一个这样的关系图$G=(P,\epsilon_1,\epsilon_2)$,其中$P$是顶点(用户,小基站,文件)的集合,$\epsilon_1$是用户与小基站相连的边的集合,$\epsilon_2$是小基站与文件相连的边的集合。关系图的一个示例如图\ref{Gamenm}所示,文件和小基站相连表示该文件被此小基站缓存,例如文件$F1,F2,F4$ 被小基站$S2$ 缓存。小基站和用户相连表示该小基站能够为此用户提供服务,例如小基站$S1$能服务用户$U1,U2$。\par
\begin{figure}{}
\centering
\includegraphics[height=7cm ,width=7cm]{Gamenm.jpg}
\caption{网络模型图}\label{Gamenm}
\end{figure}
我们定义$\mathcal{C}_n$为小基站$n$能服务的用户的集合:
\begin{equation}
\mathcal{C}_n=\{j:(n,j)\in\epsilon_1\}
\end{equation}
同时定义:
\begin{equation}
x_{in}=\begin{cases}
1&(i,n)\in\epsilon_2\\
0&(i,n)\notin\epsilon_2
\end{cases}
\end{equation}
\begin{equation}
y_{nj}=\begin{cases}
1&(n,j)\in\epsilon_1\\
0&(n,j)\notin\epsilon_1
\end{cases}
\end{equation}
其中$x_{in}=1$表示小基站$n$缓存了文件$i$,$y_{nj}=1$表示小基站$n$可以对用户$j$提供服务。那么如果一个用户$j$能获得附近小基站的服务,则有$\max_{m\in\mathcal{N}}\{x_{im}y_{mj}\}=1$,否则$\max_{m\in\mathcal{N}}\{x_{im}y_{mj}\}=0$。\par
由于用户被一个附近的小基站服务相对于被宏基站服务会有更短的延迟,获得更好的用户体验,用户会更加满意与小基站提供的服务。因此我们可以将优化问题写作如下形式:
\begin{equation}\label{Game}
\begin{split}
\max_{x_{in}}\sum_{j=1}^M\sum_{i=1}^FP_i\max_{m\in\mathcal{N}}\{x_{im}y_{mj}\}\\
s.t.\quad\sum_{i=1}^Fx_{in}\leq C,\forall n
\end{split}
\end{equation}
其中$C$是每个小基站的缓存容量大小,该目标函数表示可以从小基站下载文件的用户的数量。
\section{局部合作博弈}
在本节中,我们使用博弈论来处理上述优化问题。以分散的方式分析博弈者之间的相互作用,目的是提高系统的总体性能。博弈者根据“系统自组织”的原则选择自己的策略
\cite{prehofer2005self},这样一来博弈者之间的交互将提升系统的整体性能。\par
我们将上述问题用博弈论模型表示为$\mathcal{G}=[G,\{S_n\}_{n\in\mathcal{N}},\{U_n\}_{n\in\mathcal{N}}]$,其中$G$表示关系图,$\{S_n\}$表示博弈者(小基站)$n$的可供切换的策略集合,$\{U_n\}$ 表示博弈者$n$的效用函数。小基站的策略集为$S_n=\{\textbf{s}_n\}$,其中$\textbf{s}_n=(x_{1n},x_{2n},x_{3n},\dots,x_{Fn})^T,x_{in}\in\{0,1\}\forall i\in\mathcal{F}$,这样所有博弈者的策略可表示为一个矩阵$S=\{\textbf{s}_1,\textbf{s}_2,\textbf{s}_3,\dots,\textbf{s}_N\}\in\mathcal{S}$。接着我们可以定义除了$n$的其他所有博弈者的策略为$S_{-n}=\{\textbf{s}_1,\dots,\textbf{s}_{n-1},\textbf{s}_{n+1},\dots,\textbf{s}_N\}\in\mathcal{S}_{-n}$。\par
我们定义博弈者$n$的效用函数为可由小基站服务的用户数量,即:
\begin{equation}
U_n(S)=\sum_{j\in\mathcal{C}_n}\sum_{i=1}^FP_i\max_{m\in\mathcal{N}}\{x_{im}y_{mj}\}
\end{equation}
当用户能够被博弈者$n$的某个相邻的基站服务,即其相邻的基站已经缓存用户请求的内容文件时,那么$n$应该缓存另一个文件来提高效用函数。此外由关系图可知,每个博弈者的效用函数仅取决于其自身及其相邻基站,因此该博弈可以通过分散的方式在局部交换信息。\par
有了上述的条件,我们可以从势博弈的角度来分析分布式缓存的问题。博弈论中有一种特殊的博弈,称它为势博弈,即在博弈当中,所有的个体(博弈者)改变策略而造成的效用的改变都能被映射到全局函数上,此全局函数即称势函数,我们研究势博弈的一个重要意义在其可以帮我们分析博弈的均衡,因为所有势博弈是一定存在有纯纳什均衡(PNE)的。而在势博弈中,严格势博弈又是一种特殊的势博弈。更进一步地说,可以证明我们的分布式缓存问题是一个严格势博弈。
我们定义:如果存在一个函数$\Phi:\mathcal{S}\rightarrow R$对任意的$S\in\mathcal{S}$,都有
\begin{equation}
\forall n\in\mathcal{N},U_n(\textbf{s}_n',S_{-n})-U_n(\textbf{s}_n,S_{-n})=\Phi_n(\textbf{s}_n',S_{-n})-\Phi_n(\textbf{s}_n,S_{-n})
\end{equation}
则任意此形式的博弈$\mathcal{G}=[G,\{S_n\}_{n\in\mathcal{N}},\{U_n\}_{n\in\mathcal{N}}]$都是严格势博弈,函数$\Phi$称为博弈$\mathcal{G}$的严格势函数。下面证明我们小基站的缓存博弈是严格势博弈。\par
证明:我们定义势函数如下:
\begin{equation}
\Phi(S)=\sum_{j=1}^M\sum_{i=1}^FP_i\max_{m\in\mathcal{N}}\{x_{im}y_{mj}\}
\end{equation}
这与问题(\ref{Game})的目标函数一致。然后我们可以得出:
\begin{equation}
\begin{split}
&\Phi_n(\textbf{s}_n',S_{-n})-\Phi_n(\textbf{s}_n,S_{-n})\\
&=\sum_{j\in\mathcal{C}_n}\sum_{i=1}^FP_i\max_{m\in\mathcal{N}}\{x_{im}'y_{mj}\}-\sum_{j\in\mathcal{C}_n}\sum_{i=1}^FP_i\max_{m\in\mathcal{N}}\{x_{im}y_{mj}\}\\
&+\sum_{j\notin\mathcal{C}_n}\sum_{i=1}^FP_i\max_{m\in\mathcal{N}}\{x_{im}y_{mj}\}-\sum_{j\notin\mathcal{C}_n}\sum_{i=1}^FP_i\max_{m\in\mathcal{N}}\{x_{im}y_{mj}\}\\
&=U_n(\textbf{s}_n',S_{-n})-U_n(\textbf{s}_n,S_{-n})
\end{split}
\end{equation}
由文献\cite{monderer1996potential}可知,势函数为$\Phi$的严格势博弈$\mathcal{G}$至少有一个纳什均衡。由于满足要求的策略的数量是有限的,严格势博弈$\mathcal{G}$的每个纳什均衡都是能够从小基站下载文件的用户数量的局部最大值或全局最大值,而最佳的纳什均衡就是问题(\ref{Game}) 的全局最优解
\cite{zheng2015optimal}\cite{xu2012opportunistic}\cite{song2011optimal}。\par
我们将分布式缓存问题变换为严格势博弈问题,其最佳纳什均衡是问题的全局最优解。每个博弈者的效用被定义为可以被博弈者服务的用户数量。基于关系图,博弈者的效用函数只取决于自身及其相邻小基站。通过博弈中的博弈者之间的局部交互,可以获得整个系统的最佳性能。因而我们可以使用基于学习的博弈算法来解决主动缓存问题。
\section{分布式缓存算法}
\subsection{算法描述}
下面具体介绍我们的算法,首先我们需要定义一个博弈者的最佳响应策略$\textbf{BRS}$:博弈者$n$ 的最佳响应策略集合($\textbf{BRS}_n$)为给定其余玩家的策略来最大化博弈者$n$ 的效用的策略集合,即:
\begin{equation}
\textbf{BRS}_n=\{\textbf{s}_n^{brs1},\textbf{s}_n^{brs2},\dots\}=arg\max_{\textbf{s}_n\in S_n}U_n(\textbf{s}_n,S_{-n})
\end{equation}
由于小基站缓存的内容文件越多,它可以服务的请求就越多。而我们假设了每个小基站最多能缓存$C$ 个文件,所以有式(\ref{Game})的约束条件$\sum_{i=1}^Fx_{in}\leq C,\forall n$。 因此我们定义博弈者$n$缓存$C$个文件的策略为$\mathcal{A}_n$。\par
如前所述,我们只需要获得最佳纳什均衡,就可以实现分布式缓存问题的全局最优。由于每个博弈者只需要与其相邻博弈者交换信息,因此算法可以以分布式方式工作。
\begin{algorithm}
    \caption{分布式缓存算法}
    \begin{algorithmic}[1] %每行显示行号
       \State \textbf{Initial}:
       \State $t\gets0$
       \State $\forall n\in\mathcal{N},\textbf{s}_n(t)=(1,\dots,1,0,\dots,0)$,number of 1 is C
       \State \textbf{iteration}:
       \For{$t=1\to MAXLOOP$}
       \State 1.选择博弈者:等可能地选择一组互不干扰的小基站,记为$\mathcal{K}(t)$,
       \State 被选到的每个小基站$n\in\mathcal{K}(t)$计算当前效用函数$U_n(\textbf{s}_n(t),S_{-n}(t))$。
       \State 2.探索新策略:每个小基站$n\in\mathcal{K}(t)$独立地改变他们的策略,
       \State $n\in\mathcal{K}(t)$找到他们的最佳响应策略集合$\textbf{BRS}_n$,新策略$\hat{\textbf{s}}_n(t)$由如下方法给出
       \If{$\textbf{s}_n(t)\notin\textbf{BRS}_n$}
       \State $\hat{\textbf{s}}_n(t)$从策略集合$\textbf{BRS}_n\cap\mathcal{A}_n$中等可能地随机选择,
       \State 这样新策略既属于最佳响应策略集合又能满足存储容量的限制条件。
       \EndIf
       \If{$s_n(t)\in\textbf{BRS}_n$}
       \State $\hat{\textbf{s}}_n(t)$从策略集合$\mathcal{A}_n\setminus\textbf{s}_n(t)$中等可能地随机选择。
       \EndIf
       \State 然后每个小基站$n\in\mathcal{K}(t)$计算他们新的效用函数$U_n(\hat{\textbf{s}}_n(t),S_{-n}(t))$
       \State 3.更新策略:每个博弈者$n\in\mathcal{K}(t)$根据如下规则更新策略:
       \begin{equation}
       \begin{cases}
        Pr(\textbf{s}_n(t+1)=\hat{\textbf{s}}_n(t))=\frac{exp\{\lambda U_n(\hat{\textbf{s}}_n(t),S_{-n}(t))\}}{exp\{\lambda U_n(\hat{\textbf{s}}_n(t),S_{-n}(t))\}+exp\{\lambda U_n(\textbf{s}_n(t),S_{-n}(t))\}}\\
        Pr(\textbf{s}_n(t+1)=\textbf{s}_n(t))=\frac{exp\{\lambda U_n(\textbf{s}_n(t),S_{-n}(t))\}}{exp\{\lambda U_n(\hat{\textbf{s}}_n(t),S_{-n}(t))\}+exp\{\lambda U_n(\textbf{s}_n(t),S_{-n}(t))\}}
       \end{cases}
       \end{equation}
       \State 其中$\lambda$是学习参数,$Pr(\dots)$表示事件$(\dots)$发生的概率
       \State 所有其他的小基站保持他们的策略不变
       \EndFor
    \end{algorithmic}
\end{algorithm}
我们的算法实现求解受约束的势博弈$\mathcal{G}$的最佳纳什均衡。算法的学习参数$\lambda$在探索空间和收敛速度之间做出了权衡。$\lambda$较小表示探索策略空间较大,但收敛慢,而$\lambda$ 较大表示探索空间有限,但收敛迅速\cite{xu2012opportunistic}\cite{zheng2014optimal}。因此,我们先将$\lambda$设置为一个较小的数使得开始时有较大的搜索空间,然后随着收敛的迭代而逐渐增加。如果当算法迭代达到局部最优时,$\lambda$变得足够大(即迭代次数足够多),则最后我们将得到局部最优解。
\subsection{稳定性与最优性分析}
在本节中,我们将使用离散时间马尔科夫过程来分析所提出算法的稳定性和最优性。博弈者所有可能的策略$S$的平稳分布$\pi(S)$由下式给出:
\begin{equation}\label{Markov}
\pi(S)=\frac{exp\{\lambda\Phi(S)\}}{\sum_{S\in\mathcal{A}}exp\{\lambda\Phi(S)\}}
\end{equation}
其中$\mathcal{A}$是所有博弈者的所有可能行为的集合,$\Phi$是势函数。\par
证明:记所有博弈者在第$t$次迭代的策略为$S(t)=(\textbf{s}_1(t),\textbf{s}_2(t),\dots,\textbf{s}_n(t))$,$S(t)$是离散时间马尔科夫过程,并且该马尔可夫链是不可约且非周期的。因此,它具有唯一的平稳分布,满足如下平衡方程:
\begin{equation}
\sum_{X\in\mathcal{A}}\pi(X)Pr(Y|X)=\pi(Y)
\end{equation}
其中$X,Y\in\mathcal{A}$是任意的两个状态。下面证明由等式(\ref{Markov})给出的平稳分布满足上述平衡方程。为了方便分析,我们忽略时间参数$t$。不失一般性,我们假设状态由$X=(\textbf{s}_1,\textbf{s}_2,\dots,\textbf{s}_N)$转移至$Y=(\textbf{s}_1',\textbf{s}_2',\dots,\textbf{s}_K',\textbf{s}_{K+1},\dots,\textbf{s}_N)$,其中$\mathcal{K}=\{1,2,\dots,K\}$ 是在算法中被选中的小基站。然后我们构造序列$ X_0,X_1,X_2,\dots,X_K$,其中$X_0=X$且$\forall i\in\mathcal{K},X_i=(\textbf{s}_1',\textbf{s}_2',\dots,\textbf{s}_i',\textbf{s}_{i+1},\dots,\textbf{s}_N)$,显然$Y=X_K$。 因为每个小基站独立地改变其策略,所以有:
\begin{equation}
Pr(Y|X)=Pr(X_K|X_{K-1})Pr(X_{K-1}|X_{K-2})\cdots Pr(X_1|X_0)
\end{equation}
根据上式我们先计算$\forall i\in\mathcal{K},Pr(X_i|X_{i-1})$,显然$X_{i-1}=(\textbf{s}_i,S_{-i}),X_{i-1}=(\textbf{s}_i',S_{-i})$。在该转移概率中,我们将属于$\mathcal{A}_i$ 但不属于$\textbf{BRS}_i$ 的策略表示为$\mathcal{T}_i=\{\textbf{s}_i^1,\textbf{s}_i^2,\dots,\textbf{s}_i^T\}=\mathcal{A}_i\setminus\{\mathcal{A}_i\cap\textbf{BRS}_i\}$。 然后假设$\textbf{BRS}_i$ 中只有一个策略,则有$\textbf{BRS}_i=\{\textbf{s}_i^{brs}\}$。 当$\textbf{BRS}_i$中有多个策略时,我们可以按照同样的方法得到相同的结果。定义$X_i^{brs}=\{\textbf{s}_i^{brs},S_{-i}\}$。\par
当$\textbf{s}_i$和$\textbf{s}_i'$́同时属于$\mathcal{T}_i$时,我们有:
\begin{equation}
Pr(X_i|X_{i-1})=Pr(X_i^{brs}|X_{i-1})Pr(X_i|X_i^{brs})
\end{equation}
所以我们可以得到上面要求的转移概率为:
\begin{equation}
Pr(X_i|X_{i-1})=\begin{cases}
\frac{exp\{\lambda U_n(X_i)\}exp\{\lambda U_n(X_i^{brs})\}}{\psi_1\psi_2}&\textbf{s}_i,\textbf{s}_i'\in\mathcal{T}_i\\
\frac{exp\{\lambda U_n(X_i)\}}{\psi_3}&otherwise
\end{cases}
\end{equation}
其中$\psi_1=exp\{\lambda U_n(X_{i-1})\}+exp\{\lambda U_n(X_i^{brs})\},\psi_2=exp\{\lambda U_n(X_i)\}+exp\{\lambda U_n(X_i^{brs})\},\psi_3=exp\{\lambda U_n(X_{i-1})\}+exp\{\lambda U_n(X_i)\}$
为了简洁,我们记$\mu_i$为:
\begin{equation}
\mu_i=\begin{cases}
\frac{exp\{\lambda U_n(X_i^{brs})\}}{\psi_1\psi_2}&\textbf{s}_i,\textbf{s}_i'\in\mathcal{T}_i\\
\frac{1}{\psi_3}&otherwise
\end{cases}
\end{equation}
既而得到:
\begin{equation}
Pr(X_i|X_{i-1})=\mu_iexp\{\lambda U_n(X_i)\}
\end{equation}
假设$\mathcal{K}$被选为算法中博弈者集合的概率是$\theta$,然后有:
\begin{equation}
\begin{split}
&\pi(X)Pr(Y|X)\\
&=\frac{exp\{\lambda\Phi(X)\}}{\sum_{S\in\mathcal{A}}exp\{\lambda\Phi(S)\}}\times\theta\times\prod_{i\in\mathcal{K}}\mu_iexp\{\lambda U_n(X_i)\}\\
&=\mu exp\{\lambda\Phi(X)+\lambda\sum_{i\in\mathcal{K}}U_n(X_i)\}
\end{split}
\end{equation}
其中,$\mu=\frac{\theta}{\sum_{S\in\mathcal{A}}exp\{\lambda\Phi(S)\}}\times\prod_{i\in\mathcal{K}}\mu_i$。
同理由对称性:
\begin{equation}
\pi(Y)Pr(X|Y)=\mu exp\{\lambda\Phi(Y)+\lambda\sum_{i\in\mathcal{K}}U_n(X_{i-1})\}
\end{equation}
接下来的证明可由\cite{zheng2015optimal}中的定理4证明方法推出。
\section{算法仿真和性能分析}
在本节中,我们将我们提出的算法与上一章使用的贪心缓存(greedy caching),即所有的小基站都缓存最受欢迎的文件,进行对比。\par
我们考虑一个($1000m\times1000m$)的区域,其中小基站和用户都以均匀概率随机、独立地分布在区域中,假设每个小基站的覆盖范围半径$300m$。一种小基站与用户之间的分布关系如图\ref{gp0}所示。\par
\begin{figure}{}
\centering
\includegraphics[height=9cm ,width=12cm]{game0.jpg}
\caption{小基站与用户的分布关系示意图}\label{gp0}
\end{figure}
用户与内容文件之间的受欢迎度关系由\ref{Zipf}给出,具体的仿真参数如表\ref{Gametab}所示。
\begin{table}[h]
\centering
\begin{tabular}{*{3}{p{.33\textwidth}}}
\hline \hline
参数 & 描述 & 仿真值\\
\hline $N$ & 小基站数量 & 10 \\
       $M$ & 用户数量 & 100 \\
       $F$ & 内容文件数量 & 10 \\
       $\alpha$ & Zipf参数 &0.6 \\
       $C$ &缓存容量 &4 \\
       $r$ &小基站覆盖半径 &$300m$ \\
       $S$ &仿真区域大小 &$1000m\times1000m$\\
\hline\hline
\end{tabular}
\caption{仿真参数表}\label{Gametab}
\end{table}
下面我们比较满意的用户数随缓存容量、小基站数以及文件受欢迎度分布参数$\alpha$的变化关系。仿真时我们做$100$ 次Monte-Carlo 并求取平均值。\par
\begin{figure}{}
\centering
\includegraphics[height=9cm ,width=12cm]{game1.jpg}
\caption{满意用户数随缓存容量变化图,其中用户数100,小基站数10,文件数10,文件受欢迎度分布参数$\alpha=0.6$}\label{gp1}
\end{figure}
在图\ref{gp1}中我们将基站的缓存容量设为变量,设置其他参数为:用户的总数为100,总文件数为10,小基站数为10,文件受欢迎度分布参数$\alpha=0.6$。然后比较贪心缓存算法和博弈缓存算法的性能。性能评价指标由用户满意度来表示,即能被小基站服务的用户数。\par
由图\ref{gp1}可以看出所有算法中可以由小基站服务的用户的数量随着缓存容量的增加而增加。当小基站的缓存容量变得更大时,每个基站可以缓存更多的文件,用户可以从小基站获取的文件更多,此时用户从附近的小基站下载文件内容的概率变得更高。在所有的缓存容量情况下,我们提出的博弈缓存性能都要比贪心缓存性能好。\par
\begin{figure}{}
\centering
\includegraphics[height=9cm ,width=12cm]{game2.jpg}
\caption{满意用户数随小基站数变化图,其中用户数100,文件数10,缓存容量4,文件受欢迎度分布参数$\alpha=0.6$}\label{gp2}
\end{figure}
在图\ref{gp2}中我们将小基站数设为变量,设置其他参数为:用户的总数为100,总文件数为10,缓存容量为4,文件受欢迎度分布参数$\alpha=0.6$。然后比较贪心缓存算法和博弈缓存算法的性能。性能评价指标由用户满意度来表示,即能被小基站服务的用户数。\par
由图\ref{gp2}可以看出所有算法中可以由小基站服务的用户的数量随着小基站数量的增加而增加,并且因为小基站的部署越来越密集导致曲线性能提升越来越缓慢,曲线趋向于平坦。当小基站的密度变得更大时,每个用户可以访问更多的基站,此时用户从附近的小基站下载文件内容的概率变得更高。当小基站足够密集时,我们可以推断所有用户都可以由小基站服务。而我们提出的博弈缓存性能要比贪心缓存性能好,并且随着小基站的数量的增加,相对的性能提升会越来越明显,因为随着小基站数量的增加,贪心缓存中距离相近小基站会缓存更多重复的文件,这导致了总体缓存容量的浪费,从而其性能是较差的。而博弈缓存由于小基站数的增加,越来越多的博弈者加入进来,通过局部合作博弈,相近的基站会交换信息来提升缓存性能,整体的性能会明显地提高。\par
\begin{figure}{}
\centering
\includegraphics[height=9cm ,width=12cm]{game3.jpg}
\caption{满意用户数随文件受欢迎度分布参数$\alpha$变化图,其中用户数100,小基站数10,文件数10,缓存容量4}\label{gp3}
\end{figure}
在图\ref{gp3}中我们将文件受欢迎度分布参数$\alpha$设为变量,设置其他参数为:用户的总数为100,小基站数10,总文件数为10,缓存容量为4。然后比较贪心缓存算法和博弈缓存算法的性能。性能评价指标由用户满意度来表示,即能被小基站服务的用户数。\par
由图\ref{gp3}可以看出博弈缓存算法性能要一直优于贪心缓存算法,但随着$\alpha$的增大,贪心缓存算法和博弈缓存算法性能越来越接近,这是因为前几个很受欢迎的文件会被更多数的用户请求。只要非常受欢迎的文件被缓存,那么大多数请求都可以由小基站来服务。然而,文献[9]认为基于YouTube数据的文件受欢迎度Zipf分布中的参数$\alpha$在$0.5\sim0.6$的范围内,在这个条件下我们的博弈缓存算法的性能更好。\par
\section{小结}
本章对无线通信网络中小基站缓存文件内容的方式进行了讨论。先从博弈论的角度,证明了分布式缓存问题是一个严格势博弈,然后以此为基础设计了博弈缓存算法,来找到严格势博弈的最佳纳什均衡,也即该缓存问题的全局最优解。最后对算法进行了仿真和分析,证明了所提出的博弈缓存算法在不同的条件下,性能都要比传统贪心的缓存最受欢迎文件的算法要好,可以让更多的用户获得满意的体验。
\chapter{总结}
现在随着移动通信、移动智能设备等的发展,无线通信网络中用户越来越多地使用移动设备来体验更加丰富的在线服务,这种用户需求的增长也带来了移动互联网的数据海啸问题。
为应对这些问题,我们通过在网络边缘端设备中缓存内容来主动地服务可预测的用户需求,可以显著减少峰值回程流量,提高用户的服务体验。基于这种想法,本文主要研究了无线通信网络中的主动缓存问题。\par
我们首先设计了基于文件受欢迎度和用户与文件之间的相关度在非高峰时段主动缓存文件的缓存方案。在利用机器学习技术预测用户与文件之间的关系时,从传统协同过滤算法出发,通过对矩阵奇异值分解技术的目标函数进行不断优化,提出了经过优化后的基于奇异值分解的协同过滤算法。并以此为基础,结合迁移学习的思想,以待解决的缓存问题为目标域,以用户间D2D通信、社交网络相关的信息交换为源域,通过对源域中信息的学习来更好地预测目标域中的信息,提出了基于迁移学习的内容缓存算法。然后建立模型仿真了算法的具体性能情况,在仿真时我们首先对比了基于奇异值分解的协同过滤算法与基于迁移学习的内容缓存算法对于预测和估计文件受欢迎度矩阵的标准误差,发现基于迁移学习的内容缓存算法的误差要明显小于基于奇异值分解的协同过滤算法。然后对于缓存算法的性能,在不同的缓存容量、基站数量、CRP参数等条件下进行了仿真,得到了基于迁移学习的主动缓存算法在不同条件下都要明显优于随机缓存,已经很接近真实数据缓存的结论,证明了我们所提出算法的有效性和优越性。\par
接着讨论了主动缓存中小基站如何进行缓存的问题,并对小基站缓存内容的决策方案进行了优化。我们从博弈论角度出发,将分布式缓存问题视为一个严格势博弈问题,结合局部合作博弈提出了一个分布式博弈缓存算法,由该算法找到严格势博弈的最佳纳什均衡,也即该缓存问题的全局最优解。最后将该算法与传统的贪心缓存算法(即缓存最受欢迎的文件)对比,通过仿真,发现所提出算法对比传统方法,在不用的小基站数量、缓存容量以及文件受欢迎度分布参数等情况下都可以获得更大的用户满意度,能给用户带来更好的体验,也即拥有更好的性能。\par
目前无线通信网络中主动缓存方法的研究仍处于起步阶段,我们主要从上层的角度进行了一些探索。本文的工作还有以下几个方面可以进行扩展:考虑在用户移动的环境下,需要更智能的缓存机制,其中基站之间需要协调缓存内容以进行内容共享。研究主动缓存、干扰管理和资源调度技术的联合优化。利用组播的增益并设计更智能的编码方案来提高缓存的性能等。
\end{Main} % 结束正文

\begin{Acknowledgement}
在这次毕业设计的过程中,我遇到了很多的困难挑战,在此,我要特别向导师杨绿溪老师和帮助过我的代海波师兄、闫文师兄表示由衷的感谢。\par
本次毕业设计的很多知识是我之前的本科阶段学习中未曾接触的,我在开始做毕业设计时,在阅读的大量相关资料中没能找到头绪,杨老师和师兄一直耐心地为我答疑解惑,并指导我应该去查阅哪些文献资料。经过向他们的一次次求教,我对毕业设计的思路也逐渐清晰起来。在毕设设计的过程中,对于相关的算法代海波师兄与我交流了很多,也给我提供了很多的想法,解决了我很多具体的问题。\par
这次的毕业设计中还有许多帮助过我的人,就不一一列举了,在此谨向所有帮助过我的人致以由衷的谢意!
\end{Acknowledgement}

% 参考文献

%\bibliography{ckwx}
\bibliography{seuthesis}

%\begin{Appendix}
 % \chapter{第一个附录}

%\end{Appendix}

\newpage
\printindex % 索引

%\begin{thebibliography}{99}


%\bibliographystyle{ieee}
%\bibliography{seuthesis}

\end{document}
